% Options for packages loaded elsewhere
\PassOptionsToPackage{unicode}{hyperref}
\PassOptionsToPackage{hyphens}{url}
%
\documentclass[
]{article}
\usepackage{amsmath,amssymb}
\usepackage{lmodern}
\usepackage{iftex}
\ifPDFTeX
  \usepackage[T1]{fontenc}
  \usepackage[utf8]{inputenc}
  \usepackage{textcomp} % provide euro and other symbols
\else % if luatex or xetex
  \usepackage{unicode-math}
  \defaultfontfeatures{Scale=MatchLowercase}
  \defaultfontfeatures[\rmfamily]{Ligatures=TeX,Scale=1}
\fi
% Use upquote if available, for straight quotes in verbatim environments
\IfFileExists{upquote.sty}{\usepackage{upquote}}{}
\IfFileExists{microtype.sty}{% use microtype if available
  \usepackage[]{microtype}
  \UseMicrotypeSet[protrusion]{basicmath} % disable protrusion for tt fonts
}{}
\makeatletter
\@ifundefined{KOMAClassName}{% if non-KOMA class
  \IfFileExists{parskip.sty}{%
    \usepackage{parskip}
  }{% else
    \setlength{\parindent}{0pt}
    \setlength{\parskip}{6pt plus 2pt minus 1pt}}
}{% if KOMA class
  \KOMAoptions{parskip=half}}
\makeatother
\usepackage{xcolor}
\usepackage[margin=1in]{geometry}
\usepackage{color}
\usepackage{fancyvrb}
\newcommand{\VerbBar}{|}
\newcommand{\VERB}{\Verb[commandchars=\\\{\}]}
\DefineVerbatimEnvironment{Highlighting}{Verbatim}{commandchars=\\\{\}}
% Add ',fontsize=\small' for more characters per line
\usepackage{framed}
\definecolor{shadecolor}{RGB}{248,248,248}
\newenvironment{Shaded}{\begin{snugshade}}{\end{snugshade}}
\newcommand{\AlertTok}[1]{\textcolor[rgb]{0.94,0.16,0.16}{#1}}
\newcommand{\AnnotationTok}[1]{\textcolor[rgb]{0.56,0.35,0.01}{\textbf{\textit{#1}}}}
\newcommand{\AttributeTok}[1]{\textcolor[rgb]{0.77,0.63,0.00}{#1}}
\newcommand{\BaseNTok}[1]{\textcolor[rgb]{0.00,0.00,0.81}{#1}}
\newcommand{\BuiltInTok}[1]{#1}
\newcommand{\CharTok}[1]{\textcolor[rgb]{0.31,0.60,0.02}{#1}}
\newcommand{\CommentTok}[1]{\textcolor[rgb]{0.56,0.35,0.01}{\textit{#1}}}
\newcommand{\CommentVarTok}[1]{\textcolor[rgb]{0.56,0.35,0.01}{\textbf{\textit{#1}}}}
\newcommand{\ConstantTok}[1]{\textcolor[rgb]{0.00,0.00,0.00}{#1}}
\newcommand{\ControlFlowTok}[1]{\textcolor[rgb]{0.13,0.29,0.53}{\textbf{#1}}}
\newcommand{\DataTypeTok}[1]{\textcolor[rgb]{0.13,0.29,0.53}{#1}}
\newcommand{\DecValTok}[1]{\textcolor[rgb]{0.00,0.00,0.81}{#1}}
\newcommand{\DocumentationTok}[1]{\textcolor[rgb]{0.56,0.35,0.01}{\textbf{\textit{#1}}}}
\newcommand{\ErrorTok}[1]{\textcolor[rgb]{0.64,0.00,0.00}{\textbf{#1}}}
\newcommand{\ExtensionTok}[1]{#1}
\newcommand{\FloatTok}[1]{\textcolor[rgb]{0.00,0.00,0.81}{#1}}
\newcommand{\FunctionTok}[1]{\textcolor[rgb]{0.00,0.00,0.00}{#1}}
\newcommand{\ImportTok}[1]{#1}
\newcommand{\InformationTok}[1]{\textcolor[rgb]{0.56,0.35,0.01}{\textbf{\textit{#1}}}}
\newcommand{\KeywordTok}[1]{\textcolor[rgb]{0.13,0.29,0.53}{\textbf{#1}}}
\newcommand{\NormalTok}[1]{#1}
\newcommand{\OperatorTok}[1]{\textcolor[rgb]{0.81,0.36,0.00}{\textbf{#1}}}
\newcommand{\OtherTok}[1]{\textcolor[rgb]{0.56,0.35,0.01}{#1}}
\newcommand{\PreprocessorTok}[1]{\textcolor[rgb]{0.56,0.35,0.01}{\textit{#1}}}
\newcommand{\RegionMarkerTok}[1]{#1}
\newcommand{\SpecialCharTok}[1]{\textcolor[rgb]{0.00,0.00,0.00}{#1}}
\newcommand{\SpecialStringTok}[1]{\textcolor[rgb]{0.31,0.60,0.02}{#1}}
\newcommand{\StringTok}[1]{\textcolor[rgb]{0.31,0.60,0.02}{#1}}
\newcommand{\VariableTok}[1]{\textcolor[rgb]{0.00,0.00,0.00}{#1}}
\newcommand{\VerbatimStringTok}[1]{\textcolor[rgb]{0.31,0.60,0.02}{#1}}
\newcommand{\WarningTok}[1]{\textcolor[rgb]{0.56,0.35,0.01}{\textbf{\textit{#1}}}}
\usepackage{graphicx}
\makeatletter
\def\maxwidth{\ifdim\Gin@nat@width>\linewidth\linewidth\else\Gin@nat@width\fi}
\def\maxheight{\ifdim\Gin@nat@height>\textheight\textheight\else\Gin@nat@height\fi}
\makeatother
% Scale images if necessary, so that they will not overflow the page
% margins by default, and it is still possible to overwrite the defaults
% using explicit options in \includegraphics[width, height, ...]{}
\setkeys{Gin}{width=\maxwidth,height=\maxheight,keepaspectratio}
% Set default figure placement to htbp
\makeatletter
\def\fps@figure{htbp}
\makeatother
\setlength{\emergencystretch}{3em} % prevent overfull lines
\providecommand{\tightlist}{%
  \setlength{\itemsep}{0pt}\setlength{\parskip}{0pt}}
\setcounter{secnumdepth}{-\maxdimen} % remove section numbering
\ifLuaTeX
  \usepackage{selnolig}  % disable illegal ligatures
\fi
\IfFileExists{bookmark.sty}{\usepackage{bookmark}}{\usepackage{hyperref}}
\IfFileExists{xurl.sty}{\usepackage{xurl}}{} % add URL line breaks if available
\urlstyle{same} % disable monospaced font for URLs
\hypersetup{
  pdftitle={Earthquakes},
  pdfauthor={Samuel Richards},
  hidelinks,
  pdfcreator={LaTeX via pandoc}}

\title{Earthquakes}
\author{Samuel Richards}
\date{2023-03-22}

\begin{document}
\maketitle

Data was queried using USGS Earthquake Catalog:
\url{https://earthquake.usgs.gov/earthquakes/search/} to select all
recorded earthquakes of magnitude 4.5 and above with the following query
parameters:

\begin{itemize}
\tightlist
\item
  latitude 35.4 - 41.2
\item
  longitude 137.5 - 145.2
\item
  Timeframe(UTC): 2011-03-11 00:00:00 - 1965-01-01 00:00:00
\end{itemize}

The data was stored locally in a .csv file named \texttt{earthquakes}.

The organization also has an package called \texttt{rcomcat} to query
data directly into \texttt{R}, but its version was not compatible at the
time of this thesis.

\begin{Shaded}
\begin{Highlighting}[]
\CommentTok{\#read the data and create a new variable to be able to select annual timepoints while preserving the original timestamp}
\NormalTok{eq }\OtherTok{\textless{}{-}} \FunctionTok{read.csv}\NormalTok{(}\StringTok{"data/earthquakes.csv"}\NormalTok{) }\SpecialCharTok{\%\textgreater{}\%} 
  \FunctionTok{mutate}\NormalTok{(}\AttributeTok{timestamp =} \FunctionTok{ymd\_hms}\NormalTok{(time)) }\SpecialCharTok{\%\textgreater{}\%} 
  \FunctionTok{arrange}\NormalTok{(timestamp) }\SpecialCharTok{\%\textgreater{}\%} 
  \FunctionTok{mutate}\NormalTok{(}\AttributeTok{year =} \FunctionTok{year}\NormalTok{(timestamp))}


\CommentTok{\#subsets the data to include only the highest magnitude earthquake for each year (plus the most recent before the big one)}
\NormalTok{yearly }\OtherTok{\textless{}{-}}\NormalTok{ eq }\SpecialCharTok{\%\textgreater{}\%} \FunctionTok{group\_by}\NormalTok{(year) }\SpecialCharTok{\%\textgreater{}\%} 
  \FunctionTok{slice\_max}\NormalTok{(mag) }\SpecialCharTok{\%\textgreater{}\%} 
  \FunctionTok{distinct}\NormalTok{(year,}\AttributeTok{.keep\_all =}\NormalTok{ T) }\SpecialCharTok{\%\textgreater{}\%} 
  \FunctionTok{rbind}\NormalTok{(eq[}\DecValTok{3158}\NormalTok{,]) }\SpecialCharTok{\%\textgreater{}\%} 
  \FunctionTok{arrange}\NormalTok{(timestamp)}
\end{Highlighting}
\end{Shaded}

\hypertarget{plot-of-earthquakes}{%
\subsection{Plot of Earthquakes}\label{plot-of-earthquakes}}

The highest magnitude earthquake each year from 1965 - 2011

\hypertarget{data-preprocessing}{%
\subsection{Data Preprocessing}\label{data-preprocessing}}

\begin{Shaded}
\begin{Highlighting}[]
\CommentTok{\#Subset data to include observations from January 1, 1965 to the Greak Quake of March 11, 2011}
\NormalTok{eq }\OtherTok{\textless{}{-}}\NormalTok{ eq[}\FunctionTok{which}\NormalTok{(eq}\SpecialCharTok{$}\NormalTok{time }\SpecialCharTok{\textgreater{}=} \StringTok{"1965{-}01{-}26T23:47:37.120Z"} \SpecialCharTok{\&}\NormalTok{ eq}\SpecialCharTok{$}\NormalTok{time }\SpecialCharTok{\textless{}=} \StringTok{"2011{-}03{-}11T05:46:24.120Z}
\StringTok{"}\NormalTok{),]}

\CommentTok{\#If I want to subset by the same geographic area as the book}
\CommentTok{\#eq \textless{}{-} eq[which(eq$latitude \textgreater{} 37.72 \& eq$latitude \textless{} 38.82 \& eq$longitude \textgreater{} 141.87 \& eq$longitude \textless{} 142.87),]}

\CommentTok{\#Fine{-}tuning the geographic area of the model...}
\NormalTok{eq }\OtherTok{\textless{}{-}}\NormalTok{ eq[}\FunctionTok{which}\NormalTok{(eq}\SpecialCharTok{$}\NormalTok{latitude }\SpecialCharTok{\textgreater{}} \FloatTok{35.72} \SpecialCharTok{\&}\NormalTok{ eq}\SpecialCharTok{$}\NormalTok{latitude }\SpecialCharTok{\textless{}} \FloatTok{40.82} \SpecialCharTok{\&}\NormalTok{ eq}\SpecialCharTok{$}\NormalTok{longitude }\SpecialCharTok{\textgreater{}} \FloatTok{139.37} \SpecialCharTok{\&}\NormalTok{ eq}\SpecialCharTok{$}\NormalTok{longitude }\SpecialCharTok{\textless{}} \FloatTok{143.37}\NormalTok{),]}

\NormalTok{eq }\OtherTok{\textless{}{-}}\NormalTok{ eq[}\SpecialCharTok{{-}}\FunctionTok{as.numeric}\NormalTok{(}\FunctionTok{count}\NormalTok{(eq)),] }\CommentTok{\#to omit the 9.1}

\CommentTok{\#{-}{-}{-}data frame of magnitudes (rounded to .1) and relative frequencies}
\NormalTok{gg }\OtherTok{\textless{}{-}} \FunctionTok{data.frame}\NormalTok{(}\FunctionTok{table}\NormalTok{(}\FunctionTok{round}\NormalTok{(eq}\SpecialCharTok{$}\NormalTok{mag, }\DecValTok{1}\NormalTok{)) ) }\SpecialCharTok{\%\textgreater{}\%} 
  \FunctionTok{rename}\NormalTok{(}\AttributeTok{mag =}\NormalTok{ Var1,}
         \AttributeTok{freq =}\NormalTok{ Freq)}

\NormalTok{gg}\SpecialCharTok{$}\NormalTok{mag }\OtherTok{\textless{}{-}} \FunctionTok{as.numeric}\NormalTok{(}\FunctionTok{as.character}\NormalTok{(gg}\SpecialCharTok{$}\NormalTok{mag)) }\CommentTok{\#change to numeric rather than factors}

\NormalTok{gg}\SpecialCharTok{$}\NormalTok{freq }\OtherTok{\textless{}{-}}\NormalTok{ gg}\SpecialCharTok{$}\NormalTok{freq}\SpecialCharTok{/}\NormalTok{(}\DecValTok{2011{-}1965}\NormalTok{) }\CommentTok{\#AVERAGE annual frequencies over the 46{-}year span}


\CommentTok{\#creates a new variable representing the frequency of earthquakes of AT LEAST that magnitude}
\ControlFlowTok{for}\NormalTok{(i }\ControlFlowTok{in} \DecValTok{1}\SpecialCharTok{:}\FunctionTok{as.numeric}\NormalTok{(}\FunctionTok{count}\NormalTok{(gg)))\{}
\NormalTok{  gg}\SpecialCharTok{$}\NormalTok{freqc[i] }\OtherTok{\textless{}{-}} \FunctionTok{sum}\NormalTok{( gg}\SpecialCharTok{$}\NormalTok{freq[}\FunctionTok{c}\NormalTok{(i}\SpecialCharTok{:}\FunctionTok{as.numeric}\NormalTok{(}\FunctionTok{count}\NormalTok{(gg)))] )}
\NormalTok{\}}
\end{Highlighting}
\end{Shaded}

\hypertarget{frequency-of-earthquakes}{%
\subsection{Frequency of Earthquakes}\label{frequency-of-earthquakes}}

Relative to Magnitude

\hypertarget{linear-models-of-the-data}{%
\subsection{Linear models of the data}\label{linear-models-of-the-data}}

After log transformation of the data, a linear model is fit using
\texttt{R}.

For completeness, this model shows that on average, for every 1.0
increase in magnitude, the lapsed time between earthquakes of \emph{at
least} that magnitude is expected to increase by \(1/10^{-1.01971}\) =
10.4642956 years.

For a magnitude 9.1 earthquake, the expected frequency would be one
every \(1/10^{6.38365-1.01971(9.1)}\) = \texttt{1/10\^{}p} years.

\hypertarget{mlp}{%
\section{mlp}\label{mlp}}

Here is a conundrum. I'm using the \texttt{mlp} function to generate a
neural network to better fit the data. It's a beutiful fit actually, and
quite surprising for little data (perhaps becasue it has low noise?).
Still, I cannot seem to show that MSE \textbf{increases} as capacity
increases. It seems to be getting better\ldots{}

\begin{Shaded}
\begin{Highlighting}[]
\FunctionTok{set.seed}\NormalTok{(}\DecValTok{4723}\NormalTok{)}

\CommentTok{\#shuffle the data}
\NormalTok{df }\OtherTok{\textless{}{-}}\NormalTok{ earthquakes\_log[}\FunctionTok{sample}\NormalTok{(}\FunctionTok{nrow}\NormalTok{(earthquakes\_log)), ]}

\CommentTok{\#Extract 70\% of data into train set and the remaining 30\% in test set}
\NormalTok{train\_test\_split }\OtherTok{\textless{}{-}} \FloatTok{0.7} \SpecialCharTok{*} \FunctionTok{nrow}\NormalTok{(df)}
\NormalTok{train }\OtherTok{\textless{}{-}}\NormalTok{ df[}\DecValTok{1}\SpecialCharTok{:}\NormalTok{train\_test\_split,]}
\NormalTok{test }\OtherTok{\textless{}{-}}\NormalTok{ df[(train\_test\_split}\SpecialCharTok{+}\DecValTok{1}\NormalTok{)}\SpecialCharTok{:} \FunctionTok{nrow}\NormalTok{(df),]}

\NormalTok{mlp }\OtherTok{\textless{}{-}} \FunctionTok{neuralnet}\NormalTok{(freqc }\SpecialCharTok{\textasciitilde{}}\NormalTok{ mag,}
                 \AttributeTok{stepmax =} \FloatTok{1e+06}\NormalTok{,}
                 \AttributeTok{data =}\NormalTok{ train,}
                 \AttributeTok{hidden =} \FunctionTok{c}\NormalTok{(}\DecValTok{5}\NormalTok{,}\DecValTok{5}\NormalTok{))}

\CommentTok{\#prediction for magnitude 9.1}
\NormalTok{p }\OtherTok{\textless{}{-}} \FunctionTok{predict}\NormalTok{(mlp, }\AttributeTok{newdata =} \FunctionTok{data.frame}\NormalTok{(}\AttributeTok{mag =} \FloatTok{9.1}\NormalTok{))}
\DecValTok{1}\SpecialCharTok{/}\DecValTok{10}\SpecialCharTok{\^{}}\NormalTok{p}
\end{Highlighting}
\end{Shaded}

\begin{verbatim}
##          [,1]
## [1,] 3212.887
\end{verbatim}

Using this neural network model, the expected frequency of a magnitude
9.1 earthquake would be one every \texttt{1/10\^{}p} years.

\begin{Shaded}
\begin{Highlighting}[]
\FunctionTok{print}\NormalTok{(mlp)}
\end{Highlighting}
\end{Shaded}

\begin{verbatim}
## $call
## neuralnet(formula = freqc ~ mag, data = train, hidden = c(5, 
##     5), stepmax = 1e+06)
## 
## $response
##          freqc
## 18  0.08543020
## 25 -0.48666657
## 11  0.82438054
## 21 -0.20035983
## 29 -1.18563658
## 30 -1.36172784
## 5   1.34027164
## 8   1.06640696
## 2   1.59251467
## 7   1.16525723
## 23 -0.34053854
## 20 -0.09455611
## 26 -0.66275783
## 31 -1.66275783
## 1   1.66173040
## 24 -0.40748533
## 27 -0.96378783
## 12  0.74378235
## 16  0.33724217
## 3   1.52250093
## 14  0.53589926
## 
## $covariate
##          
##  [1,] 6.2
##  [2,] 6.9
##  [3,] 5.5
##  [4,] 6.5
##  [5,] 7.3
##  [6,] 7.4
##  [7,] 4.9
##  [8,] 5.2
##  [9,] 4.6
## [10,] 5.1
## [11,] 6.7
## [12,] 6.4
## [13,] 7.0
## [14,] 7.7
## [15,] 4.5
## [16,] 6.8
## [17,] 7.1
## [18,] 5.6
## [19,] 6.0
## [20,] 4.7
## [21,] 5.8
## 
## $model.list
## $model.list$response
## [1] "freqc"
## 
## $model.list$variables
## [1] "mag"
## 
## 
## $err.fct
## function (x, y) 
## {
##     1/2 * (y - x)^2
## }
## <bytecode: 0x000001d1a73fec30>
## <environment: 0x000001d1a73f3260>
## attr(,"type")
## [1] "sse"
## 
## $act.fct
## function (x) 
## {
##     1/(1 + exp(-x))
## }
## <bytecode: 0x000001d1a73fc408>
## <environment: 0x000001d1a7401a40>
## attr(,"type")
## [1] "logistic"
## 
## $linear.output
## [1] TRUE
## 
## $data
##    mag       freq       freqc
## 18 6.2 0.19565217  0.08543020
## 25 6.9 0.10869565 -0.48666657
## 11 5.5 1.13043478  0.82438054
## 21 6.5 0.13043478 -0.20035983
## 29 7.3 0.02173913 -1.18563658
## 30 7.4 0.02173913 -1.36172784
## 5  4.9 4.17391304  1.34027164
## 8  5.2 2.04347826  1.06640696
## 2  4.6 5.82608696  1.59251467
## 7  5.1 2.97826087  1.16525723
## 23 6.7 0.06521739 -0.34053854
## 20 6.4 0.17391304 -0.09455611
## 26 7.0 0.10869565 -0.66275783
## 31 7.7 0.02173913 -1.66275783
## 1  4.5 6.76086957  1.66173040
## 24 6.8 0.06521739 -0.40748533
## 27 7.1 0.02173913 -0.96378783
## 12 5.6 0.91304348  0.74378235
## 16 6.0 0.60869565  0.33724217
## 3  4.7 6.06521739  1.52250093
## 14 5.8 0.67391304  0.53589926
## 
## $exclude
## NULL
## 
## $net.result
## $net.result[[1]]
##              [,1]
##  [1,]  0.16116206
##  [2,] -0.63829229
##  [3,]  0.84705640
##  [4,] -0.16823733
##  [5,] -1.13932008
##  [6,] -1.26860698
##  [7,]  1.34111259
##  [8,]  1.10487184
##  [9,]  1.55611781
## [10,]  1.18600296
## [11,] -0.39904743
## [12,] -0.05614974
## [13,] -0.76088384
## [14,] -1.66378750
## [15,]  1.62315285
## [16,] -0.51765150
## [17,] -0.88532822
## [18,]  0.75629155
## [19,]  0.36910742
## [20,]  1.48678033
## [21,]  0.56751482
## 
## 
## $weights
## $weights[[1]]
## $weights[[1]][[1]]
##            [,1]       [,2]       [,3]       [,4]     [,5]
## [1,] -4.3406381 -1.0518442  0.9984048  2.4953023 6.257578
## [2,]  0.4566246  0.2508897 -0.3103534 -0.4563296 6.863085
## 
## $weights[[1]][[2]]
##            [,1]       [,2]        [,3]       [,4]        [,5]
## [1,]  0.5681135 -0.7085105 -0.30637211  0.5813091 -0.01060296
## [2,] -0.8217428 -1.8307782  0.90899699  2.6587588  3.48002847
## [3,]  1.3674441  0.9410561  1.75292506  1.6292962  0.66789457
## [4,] -3.1856985  0.9290321 -0.01800345 -1.5322427 -2.32471126
## [5,] -2.1636248  3.5366623 -3.47376583 -1.5825569 -2.64148472
## [6,]  0.8504283  0.1662884 -0.54835405 -2.5139340 -1.12437039
## 
## $weights[[1]][[3]]
##            [,1]
## [1,]  0.5047416
## [2,] -2.6428265
## [3,]  2.7920751
## [4,] -1.7627288
## [5,] -1.0273128
## [6,] -3.2349864
## 
## 
## 
## $generalized.weights
## $generalized.weights[[1]]
##             [,1]
##  [1,] -7.8654718
##  [2,]  1.1631480
##  [3,] -6.9128005
##  [4,]  5.7601565
##  [5,]  0.5274948
##  [6,]  0.4516332
##  [7,]  1.6434661
##  [8,]  7.1052890
##  [9,]  0.7878852
## [10,]  3.6235270
## [11,]  2.1057287
## [12,] 18.7103031
## [13,]  0.9220237
## [14,]  0.3005660
## [15,]  0.6514443
## [16,]  1.5228305
## [17,]  0.7508982
## [18,] -4.9900100
## [19,] -4.3629536
## [20,]  0.9740609
## [21,] -3.9439490
## 
## 
## $startweights
## $startweights[[1]]
## $startweights[[1]][[1]]
##            [,1]      [,2]       [,3]       [,4]      [,5]
## [1,] -0.9884408 0.9526854  0.8253945  0.7906610 0.5575783
## [2,]  1.4079435 0.6000437 -0.2097953 -0.4218133 1.1630847
## 
## $startweights[[1]][[2]]
##            [,1]       [,2]       [,3]       [,4]        [,5]
## [1,]  0.6704968 -0.6028092 -0.2730218  0.6146594  0.06252933
## [2,] -0.7096885 -0.2663654  0.2904974  1.3890239  0.21466900
## [3,]  1.0891604  1.5956136  1.1097431  0.9730104  0.63037064
## [4,] -0.9556536  0.7303950  0.3729143 -1.1412929  0.75107286
## [5,] -1.6397984  0.7300639 -0.8820912  1.2663787  0.92686724
## [6,]  0.9528117  0.2719898 -0.5150038 -2.4805837 -1.05123810
## 
## $startweights[[1]][[3]]
##            [,1]
## [1,]  0.3980517
## [2,] -1.0401504
## [3,]  1.4998507
## [4,] -0.4918330
## [5,] -0.5676907
## [6,] -1.9289540
## 
## 
## 
## $result.matrix
##                                [,1]
## error                  4.104001e-02
## reached.threshold      8.504801e-03
## steps                  4.473000e+03
## Intercept.to.1layhid1 -4.340638e+00
## mag.to.1layhid1        4.566246e-01
## Intercept.to.1layhid2 -1.051844e+00
## mag.to.1layhid2        2.508897e-01
## Intercept.to.1layhid3  9.984048e-01
## mag.to.1layhid3       -3.103534e-01
## Intercept.to.1layhid4  2.495302e+00
## mag.to.1layhid4       -4.563296e-01
## Intercept.to.1layhid5  6.257578e+00
## mag.to.1layhid5        6.863085e+00
## Intercept.to.2layhid1  5.681135e-01
## 1layhid1.to.2layhid1  -8.217428e-01
## 1layhid2.to.2layhid1   1.367444e+00
## 1layhid3.to.2layhid1  -3.185699e+00
## 1layhid4.to.2layhid1  -2.163625e+00
## 1layhid5.to.2layhid1   8.504283e-01
## Intercept.to.2layhid2 -7.085105e-01
## 1layhid1.to.2layhid2  -1.830778e+00
## 1layhid2.to.2layhid2   9.410561e-01
## 1layhid3.to.2layhid2   9.290321e-01
## 1layhid4.to.2layhid2   3.536662e+00
## 1layhid5.to.2layhid2   1.662884e-01
## Intercept.to.2layhid3 -3.063721e-01
## 1layhid1.to.2layhid3   9.089970e-01
## 1layhid2.to.2layhid3   1.752925e+00
## 1layhid3.to.2layhid3  -1.800345e-02
## 1layhid4.to.2layhid3  -3.473766e+00
## 1layhid5.to.2layhid3  -5.483541e-01
## Intercept.to.2layhid4  5.813091e-01
## 1layhid1.to.2layhid4   2.658759e+00
## 1layhid2.to.2layhid4   1.629296e+00
## 1layhid3.to.2layhid4  -1.532243e+00
## 1layhid4.to.2layhid4  -1.582557e+00
## 1layhid5.to.2layhid4  -2.513934e+00
## Intercept.to.2layhid5 -1.060296e-02
## 1layhid1.to.2layhid5   3.480028e+00
## 1layhid2.to.2layhid5   6.678946e-01
## 1layhid3.to.2layhid5  -2.324711e+00
## 1layhid4.to.2layhid5  -2.641485e+00
## 1layhid5.to.2layhid5  -1.124370e+00
## Intercept.to.freqc     5.047416e-01
## 2layhid1.to.freqc     -2.642827e+00
## 2layhid2.to.freqc      2.792075e+00
## 2layhid3.to.freqc     -1.762729e+00
## 2layhid4.to.freqc     -1.027313e+00
## 2layhid5.to.freqc     -3.234986e+00
## 
## attr(,"class")
## [1] "nn"
\end{verbatim}

\hypertarget{testing-capacity-levels-and-mse-measurements}{%
\subsection{testing capacity levels and MSE
measurements}\label{testing-capacity-levels-and-mse-measurements}}

\begin{Shaded}
\begin{Highlighting}[]
\FunctionTok{set.seed}\NormalTok{(}\DecValTok{4723}\NormalTok{)}
\CommentTok{\#{-}{-}{-}one hidden layer{-}{-}{-}}
\NormalTok{mlp }\OtherTok{\textless{}{-}} \FunctionTok{neuralnet}\NormalTok{(freqc }\SpecialCharTok{\textasciitilde{}}\NormalTok{ mag,}
                 \AttributeTok{stepmax =} \FloatTok{1e+06}\NormalTok{,}
                 \AttributeTok{data =}\NormalTok{ train,}
                 \AttributeTok{hidden =} \FunctionTok{c}\NormalTok{(}\DecValTok{5}\NormalTok{))}

\CommentTok{\#predictions on test data}
\NormalTok{pps }\OtherTok{\textless{}{-}} \FunctionTok{predict}\NormalTok{(mlp, }\AttributeTok{newdata =} \FunctionTok{data.frame}\NormalTok{(}\AttributeTok{mag =}\NormalTok{ test}\SpecialCharTok{$}\NormalTok{mag))}

\CommentTok{\#SSE to get test error}
\NormalTok{TestErr1 }\OtherTok{\textless{}{-}} \FunctionTok{sum}\NormalTok{((pps }\SpecialCharTok{{-}}\NormalTok{ test}\SpecialCharTok{$}\NormalTok{freqc)}\SpecialCharTok{\^{}}\DecValTok{2}\NormalTok{)}
\NormalTok{TrainErr1 }\OtherTok{\textless{}{-}}\NormalTok{ mlp}\SpecialCharTok{$}\NormalTok{result.matrix[}\DecValTok{1}\NormalTok{,]}

\FunctionTok{set.seed}\NormalTok{(}\DecValTok{4723}\NormalTok{)}
\CommentTok{\#{-}{-}{-}two hidden layers{-}{-}{-}}
\NormalTok{mlp }\OtherTok{\textless{}{-}} \FunctionTok{neuralnet}\NormalTok{(freqc }\SpecialCharTok{\textasciitilde{}}\NormalTok{ mag,}
                 \AttributeTok{stepmax =} \FloatTok{1e+06}\NormalTok{,}
                 \AttributeTok{data =}\NormalTok{ train,}
                 \AttributeTok{hidden =} \FunctionTok{c}\NormalTok{(}\DecValTok{5}\NormalTok{,}\DecValTok{5}\NormalTok{))}

\CommentTok{\#predictions on test data}
\NormalTok{pps }\OtherTok{\textless{}{-}} \FunctionTok{predict}\NormalTok{(mlp, }\AttributeTok{newdata =} \FunctionTok{data.frame}\NormalTok{(}\AttributeTok{mag =}\NormalTok{ test}\SpecialCharTok{$}\NormalTok{mag))}

\CommentTok{\#SSE to get test error}
\NormalTok{TestErr2 }\OtherTok{\textless{}{-}} \FunctionTok{sum}\NormalTok{((pps }\SpecialCharTok{{-}}\NormalTok{ test}\SpecialCharTok{$}\NormalTok{freqc)}\SpecialCharTok{\^{}}\DecValTok{2}\NormalTok{)}
\NormalTok{TrainErr2 }\OtherTok{\textless{}{-}}\NormalTok{ mlp}\SpecialCharTok{$}\NormalTok{result.matrix[}\DecValTok{1}\NormalTok{,]}

\FunctionTok{set.seed}\NormalTok{(}\DecValTok{4723}\NormalTok{)}
\CommentTok{\#{-}{-}{-}three hidden layers{-}{-}{-}}
\NormalTok{mlp }\OtherTok{\textless{}{-}} \FunctionTok{neuralnet}\NormalTok{(freqc }\SpecialCharTok{\textasciitilde{}}\NormalTok{ mag,}
                 \AttributeTok{stepmax =} \FloatTok{1e+06}\NormalTok{,}
                 \AttributeTok{data =}\NormalTok{ train,}
                 \AttributeTok{hidden =} \FunctionTok{c}\NormalTok{(}\DecValTok{5}\NormalTok{,}\DecValTok{5}\NormalTok{,}\DecValTok{5}\NormalTok{))}

\CommentTok{\#predictions on test data}
\NormalTok{pps }\OtherTok{\textless{}{-}} \FunctionTok{predict}\NormalTok{(mlp, }\AttributeTok{newdata =} \FunctionTok{data.frame}\NormalTok{(}\AttributeTok{mag =}\NormalTok{ test}\SpecialCharTok{$}\NormalTok{mag))}

\CommentTok{\#MSE to get test error}
\NormalTok{TestErr3 }\OtherTok{\textless{}{-}} \FunctionTok{sum}\NormalTok{((pps }\SpecialCharTok{{-}}\NormalTok{ test}\SpecialCharTok{$}\NormalTok{freqc)}\SpecialCharTok{\^{}}\DecValTok{2}\NormalTok{)}
\NormalTok{TrainErr3 }\OtherTok{\textless{}{-}}\NormalTok{ mlp}\SpecialCharTok{$}\NormalTok{result.matrix[}\DecValTok{1}\NormalTok{,]}

\CommentTok{\#View results}
\NormalTok{TestError }\OtherTok{\textless{}{-}} \FunctionTok{cbind}\NormalTok{(TestErr1,TestErr2,TestErr3)[}\DecValTok{1}\SpecialCharTok{:}\DecValTok{3}\NormalTok{]}
\NormalTok{TrainError }\OtherTok{\textless{}{-}} \FunctionTok{cbind}\NormalTok{(TrainErr1, TrainErr2, TrainErr3)[}\DecValTok{1}\SpecialCharTok{:}\DecValTok{3}\NormalTok{]}
\NormalTok{GeneralizationGap }\OtherTok{\textless{}{-}}\NormalTok{ TrainError }\SpecialCharTok{{-}}\NormalTok{ TestError}

\FunctionTok{data.frame}\NormalTok{(}\FunctionTok{rbind}\NormalTok{(TestError,TrainError,GeneralizationGap))}
\end{Highlighting}
\end{Shaded}

\begin{verbatim}
##                            X1          X2         X3
## TestError         0.009838334 0.008655795 0.01114061
## TrainError        0.038023503 0.017808021 0.02909132
## GeneralizationGap 0.028185169 0.009152226 0.01795071
\end{verbatim}

\end{document}
