
\subsubsection{Multi-Layer Perceptron}

Next, the \texttt{neuralnet} package \cite{neuralnet} is used to fit a multi-layer perceptron network to the earthquake data.  Several networks were tested, containing one hidden layer of differing sizes.  The function uses resilient backpropagation by default and sigmoid activation between layers.  To determine the optimal number of neurons in the hidden layer, twenty of each network was generated and the \textit{median} of each taken.  The reason for the median is because, on occasion, an iteration of a network would produce an outlandish test error and skew the results.  Therefore the median gives a more stable estimate of test error aggregate for hidden layer size.


% latex table generated in R 4.2.2 by xtable 1.8-4 package
% Wed May  3 23:45:14 2023
\begin{table}[ht]
\centering
\begin{tabular}{rlr}
  \hline
 Baseline & Test Error \\ 
  \hline
  Poisson Regression Model & 0.0394049 \\ 
  \hline
 Hidden Units & Test Error \\ 
  \hline
3 & 0.1859479 \\ 
  6 & 0.2940994 \\ 
  9 & 0.5004098 \\ 
  10 & 0.3152632 \\ 
  20 & 0.8421863 \\ 
   \hline
\end{tabular}
   \caption{Test accuracy for a Multi-layer Perceptron network of different sizes compared to baseline Poisson model.}
\end{table}

The reason that the baseline regression model performs the best in this case is attributable to two reasons:  For one, it is the simplest and has the fewest parameters to estimate.  This is desirable because of the data is so small.  Additionally, the cost function used in the model is not appropriate for the task.  By default, the \texttt{neuralnet} package uses the least squares criterion; essentially treating the model like a linear regression task.  In fact, the data had to be transformed to the logarithmic scale in order for the resilient backpropagation algorithm to converge for most cases.

Several attempts were made to replicate the networks used in (Fallah,et.al, 2009 \cite{fallah2009nonlinear}) to compare the
performance of a neural network Poisson regression model with its traditional counterpart.  Such a model would have to be amended to compete with a cost function for count data.  Particularly, based on negative log likelihood for Poisson regression, the founction would be:
$$
E_D = - \sum_{i=1}^N \left[ -\hat{y_i} + y_i log(\hat{y_i}) \right]
$$
However no successful attempt was made using the \texttt{neuralnet} package, and instead a linear regression MLP was used.
