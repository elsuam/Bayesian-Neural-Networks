The primary objective in machine learning (and therefore deep learning) is to perform well on new, unseen data.  The central challenge is finding the ideal balance between \textbf{overfitting} and \textbf{underfitting}.

\subsection{Rating Model Performance} start with \cite{Goodfellow-et-al-2016} (ch 5)

During the build process, a model is faced with a training set and measured against a test set.  With the training set, the model engages its optimization techniques (gradient decent) in order to minimiza a cost function (i.e. least squares).  The measurement of error that is reduced during optimization is known as the \textit{training error}.  When performance of new predictions is measured against the test set, the expected value of the error between predictions and actual text data is the \textit{test error} (commonly referred to as the \textit{generalization error}). \cite{Goodfellow-et-al-2016}

 When the model cannot obtain a sufficiently low test error, it is \textit{underfit}.  Often times, when a model is optimized toward a minimum training error, test error passes through a minimum rather than decreasing monotonically. \cite{mackay1992bayesian}  When the gap between the training error and test error is too great, the model is \textit{overfit}.  Somewhere between these extremes is the model's optimal \textit{capacity}.  Capacity is a model's ability to fit a wide variety of functions.  In linear regression, increasing the model's capacity would be to include polynomial terms or splines to shape the model beyond a straight line.  The same applies to neural networks as well.

\textit{(Use ggplot to display a model that is underfit and one that is overfit)  Underneath, try and recreate the Goodfellow graph of training error and test error)}

%---begin R code for ggplot display of over- and underfit---
%--If this code is modified in R, push it to GotHub to Overeaf and re-copy and paste
\hypertarget{overfit-and-underfit}{%
\subsubsection{Overfit and Underfit}\label{overfit-and-underfit}}

\begin{Shaded}
\begin{Highlighting}[]
\CommentTok{\#Generate some noisy data}
\FunctionTok{set.seed}\NormalTok{(}\DecValTok{3}\NormalTok{)}
\NormalTok{a }\OtherTok{\textless{}{-}} \FunctionTok{runif}\NormalTok{(}\DecValTok{12}\NormalTok{, }\AttributeTok{min=}\DecValTok{0}\NormalTok{, }\AttributeTok{max=}\DecValTok{90}\NormalTok{)}
\NormalTok{c }\OtherTok{\textless{}{-}} \DecValTok{24}\SpecialCharTok{*}\FunctionTok{sin}\NormalTok{(.}\DecValTok{05}\SpecialCharTok{*}\NormalTok{a) }\SpecialCharTok{+} \FunctionTok{rnorm}\NormalTok{(}\FunctionTok{length}\NormalTok{(a),}\DecValTok{0}\NormalTok{,}\DecValTok{5}\NormalTok{) }\SpecialCharTok{+} \DecValTok{40}

\CommentTok{\#create data frame and base plot}
\NormalTok{df }\OtherTok{\textless{}{-}} \FunctionTok{data.frame}\NormalTok{(a,c)}
\NormalTok{model }\OtherTok{\textless{}{-}} \FunctionTok{ggplot}\NormalTok{(df, }\FunctionTok{aes}\NormalTok{(}\AttributeTok{x =}\NormalTok{ a, }\AttributeTok{y =}\NormalTok{ c)) }\SpecialCharTok{+}
        \FunctionTok{geom\_point}\NormalTok{(}\AttributeTok{size =} \DecValTok{3}\NormalTok{, }\AttributeTok{color =} \StringTok{"purple"}\NormalTok{) }\SpecialCharTok{+}
        \FunctionTok{ylim}\NormalTok{(}\DecValTok{30}\NormalTok{,}\DecValTok{80}\NormalTok{) }\SpecialCharTok{+}
        \FunctionTok{theme\_minimal}\NormalTok{()}


\CommentTok{\# underfit model}

\NormalTok{model }\SpecialCharTok{+}
    \FunctionTok{stat\_smooth}\NormalTok{(}\AttributeTok{method =}\NormalTok{ lm,}
              \AttributeTok{formula =}\NormalTok{ y }\SpecialCharTok{\textasciitilde{}} \FunctionTok{poly}\NormalTok{(x,}\DecValTok{1}\NormalTok{),}
              \AttributeTok{fullrange =}\NormalTok{ T,}
              \AttributeTok{color =} \StringTok{"turquoise"}\NormalTok{,}
              \AttributeTok{se =}\NormalTok{ F)}

\CommentTok{\# overfit model}

\NormalTok{pol }\OtherTok{\textless{}{-}} \DecValTok{9} \CommentTok{\# polynomial order}
\NormalTok{model }\SpecialCharTok{+} 
    \FunctionTok{stat\_smooth}\NormalTok{(}\AttributeTok{method =}\NormalTok{ lm,}
              \AttributeTok{formula =}\NormalTok{ y }\SpecialCharTok{\textasciitilde{}} \FunctionTok{poly}\NormalTok{(x,pol),}
              \AttributeTok{fullrange =}\NormalTok{ T,}
              \AttributeTok{color =} \StringTok{"turquoise"}\NormalTok{,}
              \AttributeTok{se =}\NormalTok{ F)}
\end{Highlighting}
\end{Shaded}

%-----end R code snippet--------------------

$$
Side-By-Side_of_overfit_and_Underfit_Model
$$

- \textbf{Generalization}


- \textbf{Capacity} 

Including polynomials in a linear regression model increases the model's capacity, allowing it to fit the shape of the points it serves.  In neural networks...



\subsection{Addressing Model Performance}

Mention of cross-validation s a means of comparing networks trained with different parameter values \cite{mackay1992practical}

In this section, use the same equations as I had highlighted to demonstrate regularization and hint towared Bayes


- \textbf{Regularization} start with \cite{Goodfellow-et-al-2016} (ch 7) and \cite{nusrat2018comparison}

\begin{itemize}
    \item
Early Stopping
    \item
Dropout
    \item
Weight decay
\end{itemize}

\textit{Early stopping} is when the optimization algorithm is halted before the test error increases too much.  It is hoped that the algorithm stops at an optimal model capacity.

\subsection{Reconfiguring the Model}

Stochastic Neural Networks, in which the output is a random function of the input \cite{arxiv_stochast}

Earthquake example


\subsection{Quantification of Uncertainty}
 Start with:
\cite{8371683}
\cite{arxiv.1505.05424}