

Data was queried using USGS Earthquake Catalog:
\url{https://earthquake.usgs.gov/earthquakes/search/} to select all
recorded earthquakes of magnitude 4.5 and above with the following query
parameters:

\begin{itemize}
\tightlist
\item
  latitude 35.4 - 41.2
\item
  longitude 137.5 - 145.2
\item
  Timeframe(UTC): 2011-03-11 00:00:00 - 1965-01-01 00:00:00
\end{itemize}

The data was stored locally in a .csv file named \texttt{earthquakes}.

The organization also has an package called \texttt{rcomcat} to query
data directly into \texttt{R}, but its version was not compatible at the
time of this thesis.

After pre-processing, the data was fit into a frequecy table of magnitudes rounded to one decimal place.  Frequencies were calibrated to display the frequency of magnitudes at or above each earthquake size:

\begin{figure}[H]
    \includegraphics[width=0.5\linewidth]{Figures/}
    \vspace{-20pt}
    \caption{\footnotesize{The regression model to the left has minimal capacity.  It underfits the data. The model to the right depicts an overfitting model.}}
    \label{capacityviz}
\end{figure}



\begin{comment}
\hypertarget{linear-models-of-the-data}{%
\subsection{Linear models of the data}\label{linear-models-of-the-data}}
\end{comment}

After log transformation of the data, a linear model is fit using
\texttt{R}.

\begin{Shaded}
\begin{Highlighting}[]
\CommentTok{\#transforming to the log scale}
\NormalTok{earthquakes\_log }\OtherTok{\textless{}{-}}\NormalTok{ earthquakes }\SpecialCharTok{\%\textgreater{}\%} \FunctionTok{mutate}\NormalTok{(}\AttributeTok{freqc =} \FunctionTok{log10}\NormalTok{(freqc)) }

\NormalTok{linear }\OtherTok{\textless{}{-}} \FunctionTok{lm}\NormalTok{(}\AttributeTok{data =}\NormalTok{ earthquakes\_log, }\AttributeTok{formula =}\NormalTok{ freqc }\SpecialCharTok{\textasciitilde{}}\NormalTok{ mag)}

\FunctionTok{summary}\NormalTok{(linear)}
\end{Highlighting}
\end{Shaded}

\begin{verbatim}
## 
## Call:
## lm(formula = freqc ~ mag, data = earthquakes_log)
## 
## Residuals:
##      Min       1Q   Median       3Q      Max 
## -0.19955 -0.06120  0.02396  0.06852  0.16566 
## 
## Coefficients:
##             Estimate Std. Error t value Pr(>|t|)    
## (Intercept)  6.38365    0.11513   55.45   <2e-16 ***
## mag         -1.01971    0.01895  -53.80   <2e-16 ***
## ---
## Signif. codes:  0 '***' 0.001 '**' 0.01 '*' 0.05 '.' 0.1 ' ' 1
## 
## Residual standard error: 0.0956 on 29 degrees of freedom
## Multiple R-squared:  0.9901, Adjusted R-squared:  0.9897 
## F-statistic:  2894 on 1 and 29 DF,  p-value: < 2.2e-16
\end{verbatim}


For completeness, this model shows that on average, for every 1.0
increase in magnitude, the lapsed time between earthquakes of \emph{at
least} that magnitude is expected to increase by \(1/10^{-1.01971}\) =
10.4642956 years.

For a magnitude 9.1 earthquake, the expected frequency would be one
every \(1/10^{6.38365-1.01971(9.1)}\) = \texttt{1/10\^{}p} years.

\begin{Shaded}
\begin{Highlighting}[]
\NormalTok{capacity }\OtherTok{\textless{}{-}} \DecValTok{1}\SpecialCharTok{:}\DecValTok{8}
\NormalTok{prediction }\OtherTok{\textless{}{-}} \ConstantTok{NULL}

\ControlFlowTok{for}\NormalTok{(i }\ControlFlowTok{in}\NormalTok{ capacity)\{}
\NormalTok{  linear }\OtherTok{\textless{}{-}} \FunctionTok{lm}\NormalTok{(}\AttributeTok{data =}\NormalTok{ earthquakes\_log, }\AttributeTok{formula =}\NormalTok{ freqc }\SpecialCharTok{\textasciitilde{}} \FunctionTok{poly}\NormalTok{(mag,i))}
\NormalTok{  p }\OtherTok{\textless{}{-}} \FunctionTok{predict}\NormalTok{(linear, }\AttributeTok{newdata =} \FunctionTok{data.frame}\NormalTok{(}\AttributeTok{mag =} \FloatTok{9.1}\NormalTok{))}
\NormalTok{  prediction[i] }\OtherTok{\textless{}{-}} \DecValTok{1}\SpecialCharTok{/}\DecValTok{10}\SpecialCharTok{\^{}}\NormalTok{p}
\NormalTok{\}}
\FunctionTok{data.frame}\NormalTok{(capacity,prediction)}
\end{Highlighting}
\end{Shaded}

\begin{verbatim}
##   capacity    prediction
## 1        1  7.864653e+02
## 2        2  5.693638e+03
## 3        3  2.472119e+04
## 4        4  4.878898e+04
## 5        5  4.513788e-01
## 6        6  1.522366e-33
## 7        7 8.608668e-153
## 8        8  6.102735e-40
\end{verbatim}

