\chapter{Artificial Neural Networks}
Introduce ANN's more specifically here.

\section{The Architecture of Neural Networks} %-------------SECTION
This will primarily be the \textbf{BabyNeuralNet.rmd} file.

This section will be RICH in \textbf{mathematical notation}
Hint toward issues like overfitting, but elaborate more in section 2.3.

- Also, first mention \textbf{Optimization} here

\subfile{Architecture}  % Pathway to BabyNeuralNet file

%\subsection{Gradient Descent}

%\subsubsection{Simulation in R}

%\subsection{Backpropagation}

%\subsubsection{Simulation in R}

\subsection{More Optimization Algorithms and Activation Functions}

\section{Types of Neural Networks} %-------------SECTION

Write this up in R following along the code from \textbf{RAI}, with supplementary information from other sources as well.  

Obviously not all types will be covered, but here are a few.  

Include \textbf{mathematical notation} for each network mentioned.

\subsection{Multi-Layer Perceptron}
Keras Model: "sequential"

\subsection{Convolutional Neural Networks}
Unstructured image data

\subsection{Generative Adversarial Networks}
Unstructured image data

\subsection{Recurrent Neural Networks}
Unstructured text data

\subsection{Long Short-Term Memory Network}
Unstructured text data

\subsection{Convolutional Recurrent Networks}
Unstructured text data

\section{Techniques to Improve Model Performance} %-------------SECTION

The primary objective in machine learning (and therefore deep learning) is to perform well on new, unseen data. 

- \textbf{Overfitting} and \textbf{underfitting}

\subsection{Rating Model Performance}
- \textbf{Generalization}
- \textbf{Training error}  
- \textbf{Generalization error} (test error)

\subsection{Addressing Model Performance}
- \textbf{Capacity}
- \textbf{Regularization}

\subsection{Common Problems and Relevant Techniques}

Earthquake example